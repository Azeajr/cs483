\documentclass[english]{article}
\usepackage[T1]{fontenc}
\usepackage[latin9]{inputenc}
\usepackage{geometry}
\geometry{verbose,lmargin=1cm,rmargin=1cm}
\usepackage{amstext}
\usepackage{babel}
\begin{document}
\title{Homework 4}
\author{Antonio Zea Jr}
\maketitle

\section*{Problems}

\subsection{Given $\underline{\text{implementation-level description}}$ of a
Turing machine M that decides the language $A=\{w_{1}\sim w_{2}|w_{1},w_{2}\in\{0,1\}^{*}\text{ , }w_{2}\text{ is bitwise complement of}w_{1}\}$.
For example, M should accept \textquotedblleft $101\sim010$\textquotedblright{}
and reject \textquotedblleft $101\sim101$\textquotedblright . Hint:
see the Turing machine $M_{1}$ in the book}

\subsubsection*{Scan the across the tape to corresponding positions on either side
of the $\sim$ symbol to check whether these positions contain opossite
symbols. If they do not, or if no $\sim$ is found, $\emph{reject}$.
Cross off symbols as they are checked to keep track of which symbols
correspond.}

\subsubsection*{When all symbols to the left of the $\sim$ have been crossed off,
check for any remaining symbols to the right of the $\sim$. If any
symbols remain, $\emph{reject}$ ; otherwise, $\emph{accept}$.}

\subsection{Give a formal description of $M$ including a state diagram for $\delta$.\protect \\
$Q=\{q_{1},q_{2},q_{3},q_{4},q_{5},q_{6},q_{7},q_{8},q_{A},q_{R}\}$\protect \\
$\Sigma=\{0,1,\sim\}$\protect \\
$\Gamma=\{0,1,\sim,x,\_\}$\protect \\
$q_{0}=q_{1}$\protect \\
$q_{\text{Accept}}=q_{A}$\protect \\
$q_{\text{Reject}}=q_{R}$\protect \\
}
\begin{tabular}{|c|c|c|c|c|}
\hline 
 &  &  &  & \tabularnewline
\hline 
\hline 
 &  &  &  & \tabularnewline
\hline 
 &  &  &  & \tabularnewline
\hline 
 &  &  &  & \tabularnewline
\hline 
 &  &  &  & \tabularnewline
\hline 
\end{tabular}

\end{document}

